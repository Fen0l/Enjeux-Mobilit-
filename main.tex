%%%%%%%%%%%%%%%%%%%%%%%%%%%%%%%%%%%%%%%%%%%%%%%%%%%%%%%%%%%%%%%%%%%%%%%%%%%%%%%%
%2345678901234567890123456789012345678901234567890123456789012345678901234567890
%        1         2         3         4         5         6         7         8

\documentclass[letterpaper, 11 pt, conference]{ieeeconf}
\IEEEoverridecommandlockouts
\overrideIEEEmargins

\title{\LARGE \bf
Les enjeux de la mobilit\'{e} en entreprise
}

\author{Alexandre Barbot$^{1}$, Pierre Cachau$^{2}$, Anthony Pradal$^{3}$, Pierre Soulerot$^{4}$ et Timoth\'{e}e Taron$^{5}$%% 
\thanks{$^{1}$A. Barbot est un \'{e}tudiant ing\'{e}nieur sp\'{e}cialis\'{e} en informatique, au sein de l'\'{e}cole EXIA du groupe CESI, 64000 PAU, France
	}%
\thanks{$^{2}$P. Cachau est un \'{e}tudiant ing\'{e}nieur sp\'{e}cialis\'{e} en informatique, au sein de l'\'{e}cole EXIA du groupe CESI, 64000 PAU, France
	}%
\thanks{$^{3}$A. Pradal est un \'{e}tudiant ing\'{e}nieur sp\'{e}cialis\'{e} en informatique, au sein de l'\'{e}cole EXIA du groupe CESI, 64000 PAU, France
	}%
\thanks{$^{4}$P. Soulerot est un \'{e}tudiant ing\'{e}nieur sp\'{e}cialis\'{e} en informatique, au sein de l'\'{e}cole EXIA du groupe CESI, 64000 PAU, France
	}%
\thanks{$^{5}$T. Taron est un \'{e}tudiant ing\'{e}nieur sp\'{e}cialis\'{e} en informatique, au sein de l'\'{e}cole EXIA du groupe CESI, 64000 PAU, France
	}%
}

\begin{document}
\maketitle
\thispagestyle{empty}
\pagestyle{empty}

%

\begin{abstract}
Avec la multiplication des supports de communication et de travail, la question de la mobilit\'{e} en entreprise est au centre de l'attention.
La mobilit\'{e} en entreprise est une d\'{e}cision strat\'{e}gique importante et plusieurs choix s'offrent aux entreprises qui veulent prendre cette d\'{e}cision :
BYOD, CYOD, COPE, ...

Cette d\'{e}cision a un impact direct sur le syst\`{e}me d'information et sur la protection des donn\`{e}es. Aujourd'hui, beaucoup d'entreprises choississent de prendre
la solution BYOD, qui permet aux salari\'{e}s d'utiliser leur terminaux personnels pour le travail. En contrepartie, ce choix inclut de nombreux risques 
non n\'{e}gligeables (protection des donn\'{e}es, entrave dans la vie priv\'{e}e, ...). Dans cet article, nous pr\'{e}senterons le BYOD et les solutions 
qui existent pour \'{e}viter les risques cit\'{e}s pr\'{e}c\'{e}demment mais \'{e}galement les alternatives au BYOD, tel que le CYOD ou le COPE.

\end{abstract}

\section{INTRODUCTION}
Dans un monde de plus en plus connect\'{e} o\`{u} chacun se doit d’\^{e}tre joignable en continu, la fronti\`{e}re entre la sph\`{e}re priv\'{e}e et professionnelle a tendance \`{a} s'estomper de plus en plus. C'est pourquoi de nouvelles m\'{e}thodes de gestion des \'{e}quipements informatiques sont apparues. Les ordinateurs portables et les smartphones d'entreprise offrent plus de flexibilit\'{e} dans la gestion et facilitent le travail des collaborateurs au sein d'une entreprise. Trois diff\'{e}rentes m\'{e}thodes existent : le COPE, le CYOD et BYOD (1, 8). 
\linebreak

\section{TROIS MODELES DE GESTION DU MATERIEL}
\subsection{COPE (Corporate Owned Personnally Enabled)}
COPE est une m\'{e}thode classique de gestion des \'{e}quipements informatiques au sein des organisations. Dans ce mod\`{e}le, l'organisation est propri\'{e}taire de l'\'{e}quipement (Ordinateurs portable, smartphones, tablettes…) et en est donc responsable int\'{e}gralement. G\'{e}n\'{e}ralement ces \'{e}quipements d'entreprise sont \`{a} usage uniquement professionnel pour des raisons de s\'{e}curit\'{e}. 

L'avantage de ce mod\`{e}le est la simplicit\'{e} de gestion et de support sur du mat\'{e}riel g\'{e}n\'{e}ralement standardis\'{e} et maintenu gr\^{a}ce \`{a} des proc\'{e}dures adapt\'{e}es. Cette solution peut s'av\'{e}rer plus \'{e}conomique pour les employ\'{e}s (1). 
\linebreak

\subsection{CYOD (Choose Your Own Device)}
CYOD offre plus de libert\'{e} contrairement au mod\`{e}le COPE pour l'utilisateur car il peut choisir ses \'{e}quipements professionnels parmi une liste \'{e}tablie au pr\'{e}alable par l'organisation. L'entreprise est toujours propri\'{e}taire de ces \'{e}quipements et ces derniers reste \`{a} un usage exclusivement professionnel. 

Bien que la gestion et le support soient sensiblement plus complexe pour l'entreprise, il reste g\'{e}rable \'{e}tant donn\'{e} le portefeuille r\'{e}duit d’\'{e}quipements approuv\'{e}s. L'int\'{e}r\^{e}t du CYOD est d'offrir \`{a} l'utilisateur la possibilit\'{e} de personnaliser son environnement de travail pour un co\^{u}t similaire au mod\`{e}le COPE. Cette m\'{e}thode permet de supprimer tous les probl\`{e}mes juridiques et ceux li\'{e}s \`{a} la vie priv\'{e}e (1).
\linebreak

\subsection{BYOD (Bring Your Own Device)}
BYOD est le dernier des trois mod\`{e}les de gestion et le plus libre pour l'employ\'{e}. Ici, c’est l'employ\'{e} qui utilise ses propres \'{e}quipements pour son activit\'{e} professionnelle. L'entreprise n'est pas propri\'{e}taire de l'\'{e}quipement mais participe financi\`{e}rement \`{a} l'achat ou d\'{e}dommage l’employ\'{e} sur l'amortissement des \'{e}quipements. Dans ce mod\`{e}le, l'utilisateur est donc parfaitement familier avec son environnement informatique bien qu’une convention sur la s\'{e}curit\'{e} des donn\'{e}es se doit d’\^{e}tre mise en place entre lui et son employeur (2).
\linebreak

\section{CONTRAINTES}
Tous ces mod\`{e}les reposent sur les m\^{e}mes contraintes essentielles en milieu professionnel : la s\'{e}curit\'{e} des donn\'{e}es et du r\'{e}seau et la simplicit\'{e} d’utilisation. Il revient \`{a} chaque organisation de d\'{e}terminer l’emplacement du curseur entre ces deux valeurs (6) au regard de leur s\'{e}curit\'{e} et de la politique de l’entreprise. Historiquement, le mod\`{e}le COPE (7) offre une meilleure protection en isolant totalement la vie priv\'{e}e du monde professionnel. Cependant de nos jours, avec de nouveaux outils, il est possible avec le mod\`{e}le BYOD d’offrir un niveau de s\'{e}curit\'{e} \'{e}quivalent mais avec une simplicit\'{e} d’utilisation am\'{e}lior\'{e}e (3).

La simplicit\'{e} d’utilisation peut devenir complexe dans un parc h\'{e}t\'{e}rog\`{e}ne regroupant des \'{e}quipements de marque et syst\`{e}mes d'exploitation diff\'{e}rents. Cette simplicit\'{e} se doit \'{e}galement de ne pas compromettre la gestion des droits d’acc\'{e}s des utilisateurs ainsi que les documents de l'entreprise.
\linebreak

\section{OUTILS}
Aujourd’hui, cette tendance conduit les fournisseurs de progiciel \`{a} s’adapter et \`{a} trouver de nouvelles solutions techniques afin de toujours simplifier l'utilisation de leurs produits en conservant une s\'{e}curit\'{e} absolue (3).

Pour sa simplicit\'{e} d’impl\'{e}mentation et son caract\`{e}re critique pour les utilisateurs, les services de mails et de CRM sont souvent les premiers \'{e}l\'{e}ments \`{a} b\'{e}n\'{e}ficier d’une d\'{e}marche BYOD. Cela permet aux utilisateurs d’avoir acc\`{e}s aux services depuis leurs t\'{e}l\'{e}phones personnels (5).

Les MDM (Mobile Device Management) sont des m\'{e}thodes pour accompagner les DSI dans cette migration de l’information en permettant une gestion centralis\'{e}e de tous les \'{e}quipements. 

Les poids lourds du secteur sont aujourd’hui incarn\'{e}s par VMware et son produit AirWatch, la solution propos\'{e}e par MobileIron ou bien encore celle de Citrix, XenMobile. A eux seuls, ils repr\'{e}sentent environ 58\% du march\'{e} des MDM (4). 

Dans la suite du classement, on trouve de nombreux MDM propri\'{e}taires des compagnies d’appareils mobiles tels que Apple, Samsung, BlackBerry mais aussi Microsoft, IBM et Cisco. 

En compl\'{e}ment du MDM, une sur-couche MAM (Mobile Application Management) pourrait id\'{e}alement \^{e}tre appliqu\'{e}, afin de proposer une gestion ax\'{e}e sur les applications respectant la vie priv\'{e}e (notamment concernant la m\'{e}thode BYOD). De l'installation \`{a} la configuration, le MAM s’occupe de la gestion du t\'{e}l\'{e}phone au niveau applicatif et non au niveau du terminal t\'{e}l\'{e}phonique. 

Chacun de ces outils permet, avec plus ou moins de simplicit\'{e}, de g\'{e}rer des \'{e}quipements mobiles afin de leur appliquer la politique informatique de l'entreprise. On peut revenir par exemple sur des utilitaires tel que AirWatch de VMware ou XenMobile de Citrix qui sont tout particuli\`{e}rement adapt\'{e}s \`{a} une d\'{e}marche BYOD en offrant par exemple une conteneurisation des processus d’entreprise ainsi qu’un service de VPN permettant une protection des donn\'{e}es d’entreprise.
\linebreak

\section{CONCLUSION}
Dans un contexte d’optimisation des co\^{u}ts, de d\'{e}veloppement de la mobilit\'{e} et d’un monde de plus en plus connect\'{e}, les entreprises prennent et devront prendre des d\'{e}cisions strat\'{e}giques sur la fa\c con de g\'{e}rer leur parc informatique. Le BYOD, CYOD et COPE ainsi que leurs d\'{e}riv\'{e}s offrent donc des mod\`{e}les de r\'{e}ponse adaptable \`{a} chaque organisation pour coller aux mieux aux exigences fonctionnelles. Aujourd'hui, l'opportunit\'{e} qui s’ouvre aux entreprises est de bien consid\'{e}rer chacune des options avec en perspective la croissance toujours plus effr\'{e}n\'{e}e du nombre de p\'{e}riph\'{e}riques par utilisateur (Laptop, smartphone, tablette, objet connect\'{e}…) avec les impacts induits sur le SI.

\addtolength{\textheight}{-12cm}

\begin{thebibliography}{99}
\bibitem{c1} 01NET.com, BYOD, CYOD, COPE : qu'est-ce que c'est ? Octobre 2017
\bibitem{c2} Panoptinet, C’est quoi le BYOD ? Septembre 2012
\bibitem{c3} Matt Grech, The State of BYoD in 2017: How to Secure Your Security Nightmare; Mars 2017
\bibitem{c4} IngramMicroAdvisor, 23 BYOD Statistics You Should Be Familiar With; Mai 2015
\bibitem{c5} Michael Lazar, BYOD Statistics Provide Snapshot of Future; Janvier 2017
\bibitem{c6} Zoran Mitrovic, Ivan Veljkovic, Grafton Whyte, Kevin Thompson, Introducing BYOD in an organisation: the risk and customer services viewpoints; Novembre 2014
\bibitem{c7} Dave Snow, BYOD Versus Corporate-Liable: How do you COPE? Mars 2014
\bibitem{c8} Le BYOD en entreprise; Janvier 2016
\end{thebibliography}

\end{document}
